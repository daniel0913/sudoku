\documentclass[english,a4paper]{article}
\begin{document}
\title{Project Report}
\author{Edon Kelmendi}
\maketitle
\newpage
\section{Introduction}

Sudoku is a combinatoral number placement puzzle. It is usually played
in a $9\times 9$ grid. That is a grid which contains 9 columns and 9 rows,
and whose elements are called cells. Grids also contain 9 blocks which
are smaller grids of size $3\times 3$ that do not overlap with each other.\\

Some of the cells might be filled with numbers from 1 to 9 and others
might not be. The ones that are already filled are called hints.\\

The purpose of the game is to fill the unfilled cells with numbers
from 1 to 9 such that no two numbers are repeated in the same row,
column or block. Finding a solution to this puzzle is known to be an
\emph{NP} problem.\\ 

In this project the size of the grid to be solved/generated is
generalized but a constant is used as a higher bound (by default set
to 64).\\

\section{Heuristics}

\subsection{Locked candidates}

One of the heuristics implemented on this project besides the
cross-hatching and lone number one, is the one called \emph{removal of
  the locked candidates}.\\

Since we have the requirement that the numbers inside a block should
not repeat this heuristic looks for a number in a block which is to be
seen only on one row (column) and removes it from the rest of the row
(column). Since no matter the state of the grid at later points in
time that certain element is going to be in that certain block at that
certain line (column).\\

This was done using two functions \textbf{rm\_locked\_candidates} which
takes as one of the parameters the number of the block on which we
want to look (so it must be run over all blocks), and senses if there
is a candidate locked in a row or column that, of course, is not a
singleton. If it finds one (or more) it calls the
\textbf{cross\_off\_candidate} function to remove the locked candidates
from the rest of the row or column.\\

Therefore the job of \textbf{cross\_off\_candidate} is to remove some
set of colors from the cells in the row (column) but not from the
certain block.\\

This heuristic makes the program less efficient in smaller grids but,
with it the program performs better in bigger grids. And since smaller
grids are solved relatively fast anyway, in general, such a heuristic
seems to be helpful. \\

\subsection{Naked set}

The other heuristic implemented is the \emph{naked set} one. The gist
of this heuristic is this observation: if we can find in a subgrid
\emph{n} colors of cardinality \emph{n} which are all equal to one
another then it is safe to cancel all the elements of such a set from
other cells of a subgrid.\\ 

A subgrid is a row, column, or block.\\

This heuristic is implemented in the function \textbf{naked\_set},
which in contrast to the locked candidates heuristic can be mapped
over all the subgrids of a grid, same as with other heuristic
functions. This functions helper is \textbf{rm\_naked\_set} which
cancels the colors of a naked set from a given subgrid while leaving
the ones that are first encountered.

\section{Grid Generation}

Generating a grid is harder, computationaly, since (at least in this
case) it involves solving the grid a number of times.\\

\subsection{Unstrict grid}

Generating an unstrict grid of a certain size involves solving an
empty grid of that size, and at points where we have to make guesses,
the guesses that we make should be random. For performance reasons we
choose by random from the cells with smallest cardinality.\\

This way we get a new randomly generated but already solved grid. Then
we just take out 2/3 of the cells. This, of course, does not guarentee
that there will be only one solution, but just that it will be at
least one solution.\\

\subsection{Strict grid}

We do the same as with the unstrict grid, up to the point of cleaning
off the 2/3 of the cells, because now we need a unique solution.\\

We could have just started removing cells one by one, and always
checking if the grid is solvable using only heuristics, and if no we
would stop. But this doesn't mean that we can't remove more numbers
from cells and still have a unique solution. And such a thing would
lead to boring puzzles with lots of hints.\\

So the road that we take is this: we define a function
\textbf{num\_of\_solutions}, which does what it name implies, it returns
the number of solutions that a given grid has. And then generating a
strict grid is reduced to removing numbers from cells randomly and
calling this function on the resulting grids to see if the return
value has changed from 1. When it does, the step before hand will be
the generated strict grid.\\

Note that finding lowest number of hints required such that the grid
has a unique solution is still an open problem. This algorithm will
just generate grids with different numbers of hints.\\

It also should be noted that in this specific implementation the seed
value for random number generation is taken as the number of seconds
past the epoch. So one can't generate two different grids in one
second, they will be the same.

\section{Implementation}

The whole program is divided into five modules: 
\begin{enumerate}
\item \textbf{main}: Holds the main function and closely related
  functions to parsing command line parameters such as \textbf{usage}
  etc.
\item \textbf{parser}: Holds the parser for the grids that are given
  as input and other helper functions, mostly for handling parser
  errors.
\item \textbf{preemptive\_set}: Implementation of preemptive set data
  structure, in our case a 64 bit bitfield but such a fact is well
  hidden by the interface, which exports all the neccessary operations.
  Can be used as a stand-alone library.
\item \textbf{heuristics}: Implements the above mentioned heuristics
  for solving the sudoku puzzles, together with a set of a much needed
  helper functions.
\item \textbf{sudoku}: Holds both the sudoku grid solver algorithm and
  the generation one, has quite loose coupling with the heuristics
  module just calls one function defined in it. 
\end{enumerate}

Optimizations taken were simple ones, such as checking if a grid is
consistent without doing too many steps with heuristics saved a bit of
time.\\ 

Tried making the above mentioned modules as loosely coupled as
possible, in such a way that there is a very small interface between
them and in the future the parser or euristic module can be changed
efficiently.

\section{Code quality}

Testing the program on lots of grids was chosen as a way to ensure the
quality and robustness of the program. This task was relieved by the
generation part (we could now generate new test cases), and a huge set
of unsolved nine by nine grids. On most of these tests (as long as it
was practically feasible) valgrind was run to ensure that memory leaks
are not possible.\\

Testing on grids of size $9\times 9$ was more practical because if we
would see an unwanted behaviour it would be easier to track where the
program went wrong just because of the size of the grid. While on
smaller grids the possibility of problems arising was smaller and on
larger grids it was too hard to see where the problem is arising.\\

Another reason why the test were usually made on $9x9$ grids is that
we had a big set of grids which we knew to be solvable taken from:\\
\emph{http://mapleta.maths.uwa.edu.au/~gordon/sudokumin.php}\\
.These are 49151 distinct sudoku $9x9$ grids with 17 hints.\\

The format on which this set of grids is given can't be parsed by our
parser module, but they can easily be turned to one a job which the
script \textbf{pandg.rb} does (careful when you run it, it will spam
your working directory with 49151 files named \emph{grid-n} where $n$
is the $n$-th grid. One can then run the program on these generated
files.\\

Other measures were taken on ensuring the robustness of the program
such as trying to make functions less depended on one another or
trying to make this dependence more explicit, qualifying variables
with const when we're sure that it won't be changed, and other
commonly used rules that help. 

\end{document}
